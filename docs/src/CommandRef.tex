\section{Commands}

\subsection{Movement and Attacking}

Move using number keys, vi keys, or arrow keys.  Attempting to move into a
creature's tile will either attack it, if the creature is hostile, or
else prompt for confirmation to attack, switch places, or squeeze past.
If carrying a digging implement, and trying to move into a diggable tile
(e.g. rock, earth), an attempt to dig will be made.

\begin{center}
\begin{tabular}{|c|c|c|}
\hline
7, y & 8, k, $\uparrow$ & 9, u \\
\hline
4, h $\leftarrow$ & s, . & 6, l, $\rightarrow$ \\
\hline
1, b & 2, j $\downarrow$ & 3, n \\
\hline
\end{tabular}
\end{center}

\begin{description}
\item[w]
Automatic movement
\item[$\textless$] 
Ascend on world map tiles, staircases, etc.
\item[$\textgreater$] 
Descend on world map tiles, staircases, etc.  On a non-staircase tile,
while carrying a shovel, dig.
\end{description}

\subsection{Actions}

\begin{description}
\item[5]
Rest for a while.
\item[,] 
Pick up item.
\item[;]
Pick up all items.
\item[d] 
Drop item.
\item[i] 
View equipment, items, and the item codex.
\item[I]
View items and the item codex.
\item[\_] 
Pray.
\item[s]
Search your surroundings.
\item[x] 
Examine any tile currently in view.
\item[f] 
Fire a missile.
\item[D] 
Drink a potion.
\item[r] 
Read a scroll or a book.
\item[\$] 
Display number of ivory pieces currently on hand.
\item[e] 
Eat a comestible.
\item[C] 
Chat with an adjacent creature.
\item[a] 
Apply a nearby terrain feature.
\item[N]
Inscribe on the current tile, or name the current part of the world.
\item[q] 
List current quests.
\item[z] 
Cast (`z'ap) a known spell.
\item[B] 
Display bestiary information.
\item[v] 
Evoke a wand.
\item[K]
Kick.
\item[O]
Offer a sacrifice while at an altar.
\item[/]
Skin a corpse.
\item[P]
Show the current piety level.
\item[X]
Show experience-related details.
\item[c]
Search the item codex.
\end{description}

\subsection{The Bestiary and Item Codex}

The bestiary and item codex provide a way to get more information about the creatures and items you encounter in your journey.  The bestiary is a compendium of information on creatures throughout the world, providing a verbose description of every beast and person you might meet.

The item codex is used to get information on items.  These can be items held in your possession, but the codex can also be used to search for items you do not have - artifacts, weapons, armour, wands, and anything else you might find.  The item codex provides more information than the bestiary, showing resistances and enchantments, details specific to certain types of item (nutrition for food and potions, evade and soak for armour, charges for wands, and so on), and also provides a textual description of the item itself.

\subsection{Leaving the Game}

\begin{description}
\item[S] 
Save the game, then quit.
\item[Q] 
Quit the game without saving -- the current character will be lost.
\end{description}

\subsection{Miscellaneous}

\begin{description}
\item[M]
View messages in the message buffer.
\item[W] 
Melee weapon information.
\item[R] 
Ranged weapon information.
\item[V] 
Game version details.
\item[T] 
Current date, time, and weather.
\item[!]
Toggle autopickup.
\item[@]
Display character details.
\item[\#] 
Dump character details to disk.
\item[F1] 
Show current resistance information.
\item[F2] 
Show current conduct information.
\item[F3]
Show skills.
\item[F9]
Switch colour palettes, if the display type allows it.
\item[F10]
Switch between tiles and ASCII, if applicable.
\item[F11] 
Execute a Lua statement.
\item[F12] 
Reload Lua scripts, textures, and string identifiers.
\end{description}

