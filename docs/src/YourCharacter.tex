\section{Your Character}

Characters in {\it Shadow of the Wyrm} are created by selecting a race,
class, and deity.  The starting race and class are important, and have a
large impact on game play. They determine starting statistics, skills, and 
equipment, and affect the difficulty of the game.  Some races and classes
have options not available to others: fae don't have a hunger clock;
witchlings get free primordial spell castings with each level; pugilists
can do an increasing amount of unarmed damage as they go.  

The choice of deity will one day become important.  Currently, the only
real difference in deities is that some of them dislike certain things
(cannibalism, certain forms of desecration, etc) and have different lists
of artifacts they can grant.  

\subsection{Races}

The races of {\it Shadow of the Wyrm} range from diminutive to massive, and
each race has its own strengths and weaknesses.  Some make great warriors, 
while some are better suited for magic or thievery; there are others that
are more neutral and can work with any class.  The races of {\it Shadow
of the Wyrm} are listed below.

\begin{itemize}
\item {\bf Humans} live throughout the world, in the centre lands, on
islands large and small, making their homes in cities and settlements, 
outposts and camps.  They are short and tall, fair and dark, settled and 
nomadic, but always adaptive, always resourceful, and always able to make 
the most of any situation. \textbf{Initial Skills}: Boating, Carrying,
Detection, Fishing, Foraging, Hiding, Swimming, Short Blades, Bludgeons,
Daggers, Spears, Rocks, Slings.

\item {\bf Wood Elves} retreated into the forests millennia ago, building 
their civilizations in the treetops and branches.  Within the woods, they 
blend in with their surroundings, their clothing dyed deep shades of green 
and brown.  Wood Elves are skilled archers, and are known for their 
devotion to swordsmanship. \textbf{Initial Skills}: Archery, Awareness,
Carrying, Detection, Forest Lore, Herbalism, Hunting, Night Sight, Swimming,
Long Blades, Bows.

\item {\bf Mountain Elves} live high up in the mountains, far above the 
rest of the world.  They are stockier than other elves, and even many 
humans, fortified by a life in the thin, cold air.  Mountain Elves 
typically wear animal skins, and craft weapons out of the tools at hand, 
making use of stone, animal bones, and the little wood at hand.
\textbf{Initial Skills}: Archery, Awareness, Carrying, Detection, 
Mountain Lore, Mountaineering, Night Sight, Spelunking, Swimming,
Bludgeons, Rocks, Slings.

\item {\bf Snakelings} are the product of black magic, wielded by some
unknown sorceror many thousands of years ago to fuse man and serpent. 
The resultant creatures horrified the world: standing nearly as tall as men,
the creatures have the scales and head of an enormous snake, with the 
remaining features human in appearance.  Cast to the fringes of the world, 
Snakelings band together in fens and marshes, hunting under darkness with 
barbed spears and javelins.  \textbf{Initial Skills}: Awareness, Boating,
Carrying, Detection, Marsh Lore, Night Sight, Swimming, Spears, Thrown
Spears.

\item {\bf Fae} are the remnants of the so-called faerie folk that lived in
the world far before the elves.  They require neither food nor drink, but 
enjoy both, holding great feasts deep within ancient forests.  Time has 
brought down the Fae, and few remain.  Those that do tend to travel 
together in caravans through the most inhospitable of terrain, keeping 
their company close and blocking out the rest of the world.  Fae are 
stealthy tricksters, and tremendously skilled with magic. 
\textbf{Initial Skills}: Awareness, Carrying, Detection, Escape, Forest
Lore, Herbalism, Hunting, Magic, Night Sight, Stealth, Daggers.

\item {\bf Dwarves} live deep within mountains, far below the 
naturally-occurring caverns and caves that serve as the entranceways.  They
carve their cities from the rock itself, and mine the mountains for its 
ore.  They are short, about two-thirds the size of men, but stocky, and 
disproportionately strong.  Dwarves are skilled with axes, hammers and 
crossbows. \textbf{Initial Skills}: Carrying, Detection, Disarm Traps,
Mountain Lore, Night Sight, Smithing, Spelunking, Axes, Bludgeons,
Crossbows.

\item {\bf Goblins} are short, cunning humanoids with sharp features and 
sharper teeth.  They live in small tribes on the outer islands, though 
recently they have been spotted closer and closer to the centre lands.   
Despite their smaller size, they are ferocious creatures, afraid of 
nothing.  They wear ratty, tattered clothing, and are skilled with their 
sharp, curved blades. \textbf{Initial Skills}: Carrying, Detection,
Disarm Traps, Dungeoneering, Escape, Hiding, Stealth, Swimming, Short
Blades, Thrown Daggers, Slings

\item {\bf Ogres} are huge and thickly-built, with the strength of many 
men.  Though they are often seen with goblins, and share those creatures 
fierce nature, they are not nearly as intelligent.  Ogres make up for this 
with a tremendous strength and fortitude.  They are often seen wearing 
scraps of armour plundered from their kills, and prefer great clubs and 
hammers above all else. \textbf{Initial Skills}: Carrying, Combat,
Detection, Foraging, Hunting, Intimidation, Skinning, Tanning, Bludgeons.

\item {\bf Giants} are even older than the elves, and warred with the 
faerie folk when the world was still young.  Massive and tremendously 
strong, they stand twice as tall as the tallest man.  They live in the 
outer reaches of the world, where their settlements are constantly under 
attack by goblins and ogres, whom they hate. Giants prefer large, stone 
clubs, though when these are unavailable, a small tree will often do.
\textbf{Initial Skills}: Carrying, Combat, Detection, Intimidation,
Mountain Lore, Bludgeons, Rods and Staves, Thrown Bludgeons.

\item {\bf Gnomes} are tiny creatures that live in caverns and caves, and 
have since recorded time.  They stand even shorter than dwarves, and are 
much less strong.  They are intelligent and quick; gnomes love riddles and 
puzzles, and have a deep affinity for magic.  In combat, they prefer knives,
daggers, and small swords. \textbf{Initial Skills}: Awareness, Carrying,
Detection, Disarm Traps, Dungeoneering, Literacy, Medicine, Mountain Lore, 
Spelunking, Wandcraft, Short Blades, Daggers, Rocks.
\end{itemize}

In addition to these races, there are a number of additional races not
available to player characters: animal, construct, demon, divine, dragon,
humanoid, insect, jelly, monster, plant, undead, and spirit.

\subsection{Classes}

A character's class represents its background or occupation.  There are
many possibilities: powerful warriors, mighty wizards, holy pilgrims; but
also skilled smiths and artisans, street-hardened pugilists, and the
ubiquitous adventurer.  Class influences a character's starting statistics
and equipment, and grants its own set of initial skills.

\begin{itemize}
\item {\bf Adventurers} travel the world, seeking to make money and a name 
for themselves.  While they lack the raw strength of Warriors, the stealth 
and cunning of Thieves, or the great learning of Wizards, they nonetheless 
have characteristics of all three, and are well-prepared for the challenges
that lie ahead.  \textbf{Initial Skills}: Awareness, Boating, 
Cantrips, Dungeoneering, Escape, Fishing, Foraging, Hiding, Jumping, 
Literacy, Spelunking, Stealth, Short Blades, Long Blades, Bludgeons, 
Daggers, Spears.

\item {\bf Apothecaries} prepare potions, salves, and remedies, assisting 
physicians by providing their medical materials.  This training gives 
apothecaries a practical knowledge of medicine and healing.  In addition, 
many dabble in hedge magic and cantrips, providing some practical magical 
knowledge while remaining focused on the duties of their trade.
Apothecaries start with a complete knowledge of potions.
\textbf{Initial Skills}: Bargaining, Brewing, Cantrips, Carrying,
Herbalism, Literacy, Medicine, Papercraft, Scribing, Bludgeons.

\item {\bf Archers} are combatants skilled with ranged weapons such as 
bows, slings, and crossbows.  While others can also learn these weapons 
effectively, Archers can maintain their accuracy over much greater 
distances.  Though they possess some of the skill of Warriors in close 
combat, their focus is on ending the danger before it gets too close.  
\textbf{Initial Skills}: Archery, Awareness, Bowyer, Detection, Escape,
Fletchery, Hiding, Jumping, Daggers, Bows, Crossbows, Rocks, Slings,
Thrown Spears.

\item {\bf Artisans} are creative folk who craft jewellery, clothing, and 
other useful items.  With training, they can focus their efforts and craft 
truly spectacular creations.  Artisans learn their trade after apprenticing
to a master crafter for many years, and with their training complete, are
finally ready to make their way.  \textbf{Initial Skills}: Bargaining,
Cantrips, Carrying, Crafting, Detection, Jeweler, Literacy, Papercraft, 
Scribing, Skinning, Tanning, Weaving, Daggers.

\item {\bf Merchants} travel from place to place, always looking to sell 
their wares.  In their travels, they learn a great many things about their 
merchandise, and are experts at identification.  Each merchant begins their
travels with a full understanding of all items.
\textbf{Initial Skills}: Awareness, Bargaining, Cantrips, Carrying,
Detection, Herbalism, Jeweler, Literacy, Lore, Bludgeons, Whips.

\item {\bf Minstrels} are itinerant musicians.  They are instrumentalists 
and singers, learned in the rudiments of almost any form of music.  Others 
find them quite charismatic and charming.  Minstrels are often able to stir
and sway their audiences on the strength of a particularly good performance.
\textbf{Initial Skills}: Awareness, Detection, Hiding, Leadership, 
Literacy, Lore, Music, Scribing, Stealth, Daggers.

\item {\bf Nobles} are born of privilege and money.  Lords and Ladies, 
Kings and Queens, Thanes and Chiefs; all are examples of the fortunate 
nobility.  A life free from hard labour allows nobles to focus their 
efforts on leadership and military training, to better lead their people 
to victory.  \textbf{Initial Skills}: Boating, Detection, Escape, 
Herbalism, Intimidation, Leadership, Literacy, Religion, Swimming, 
Short Blades, Long Blades, Whips, Bows, Crossbows.

\item {\bf Oracles} possess a keen sense of the future and present, and can
see things that others can't: happiness, true love, long life; but also 
hexes, illnesses, and death.  As favoured creatures of fate, they are 
immune to bad luck and curses.  Their predictions have an almost unfailing 
accuracy, which causes Oracles to be both respected and feared.
\textbf{Initial Skills}: Awareness, Blind Fighting, Cantrips, Detection,
Literacy, Magic, Night Sight, Papercraft, Religion, Scribing, Rods and
Staves, Mystic Magic.

\item {\bf Pilgrims} are devout followers and messengers of the divine.  
They seek enlightenment through travel, devotion, and prayer.  They carry 
their life on their back, travelling from place to place, rarely staying 
long.  Their piety gives them a sixth sense about objects, allowing them
to avoid the burning cold of cursed items; and they are favoured by the Nine, 
who grant them the ability to learn the magic of divine mysteries.
\textbf{Initial Skills}: Awareness, Detection, Dungeoneering, Herbalism,
Literacy, Lore, Magic, Medicine, Religion, Scribing, Swimming, Bludgeons, 
Spears, Divine Magic.

\item {\bf Pugilists} are fighters who focus solely on unarmed combat, 
relying on their fists and instincts to get them out of trouble.  
Quick-witted and nimble, they eschew the trappings of any armour heavy 
enough to hinder their movements.  Though they are skilled at all forms of
unarmed combat, their skill with their bare fists increases as they gain
more experience.  \textbf{Initial Skills}: Awareness, Blind Fighting, 
Combat, Detection, Dungeoneering, Foraging, Intimidation, Stealth, 
Swimming, Unarmed.

\item {\bf Rovers} are solitary figures who spend the majority of their 
lives in the wild.  At home far from civilization, rovers can be found in 
the highest peaks, deepest forests, and darkest caves.  They spurn a life 
of comfort and prefer instead a life outdoors, surviving in the most 
inhospitable parts of the world.  \textbf{Initial Skills}: Awareness,
Beastmastery, Bowyer, Desert Lore, Detection, Dual Wield, Fishing, 
Fletchery, Foraging, Forest Lore, Herbalism, Hiding, Hunting, Leadership, 
Marsh Lore, Mountain Lore, Mountaineering, Skinning, Swimming, Tanning, 
Axes, Bows, Rocks, Slings.

\item {\bf Sages} are scholars.  They study the seen and unseen to the 
exclusion of all else.  Sages are thin and frail.  Living alone, they often
withdraw from society to devote their lives to study.  As they increase 
their knowledge and edge closer to enlightenment, they gain access to the 
arcane and divine, the mystic and the primordial.  \textbf{Initial Skills}:
Cantrips, Detection, Herbalism, Literacy, Lore, Magic, Medicine, Papercraft,
Religion, Rods and Staves, Scribing, Wandcraft, Weaving, Arcane Magic, 
Divine Magic, Mystic Magic, Primordial Magic.

\item {\bf Seafarers} are drawn to the open water.  They make their living 
travelling the rivers and oceans, spurning a life on land.  The weather 
hardens them, and they are better able to withstand the rigors of cold.  
Fishers, pirates, and sailors are all examples of Seafarers.
\textbf{Initial Skills}: Awareness, Boating, Combat, Detection, Fishing,
Foraging, Marsh Lore, Oceanography, Skinning, Swimming, Tanning, Spears.

\item {\bf Shepherds} are among the weakest and most lowly in society.  
Tending to their flocks, or those of others, they live at the fringes of 
civilization, ekeing out a living from the land.  They are not strong in 
combat, nor with magic, but some have said they enjoy a special status 
with the divine.  \textbf{Initial Skills}: Awareness, Detection, Fishing,
Foraging, Herbalism, Religion, Skinning, Swimming, Tanning, Rods and 
Staves.

\item {\bf Smiths} are artisans of iron and steel.  Using their bellows, 
hammers, and anvils, they are able to improve upon weapons and armour, 
able to turn the most mediocre example into a masterpiece.  From spending 
a lifetime in front of the forge, they are bothered little by great heat.
\textbf{Initial Skills}: Bargaining, Carrying, Combat, Detection, 
Fishing, Intimidation, Mountain Lore, Skinning, Smithing, Tanning,
Bludgeons.

\item {\bf Thieves} are nimble and fleet of foot.  They specialize in the 
redistribution of wealth, either to themselves, or to others.  Loosely 
organized into guilds, they learn early on how to pick pockets, open 
locks, scale walls, and deal with traps.  It is said that they are so
stealthy that their nefarious activities, such as graverobbing, are hidden
from the Nine themselves.  \textbf{Initial Skills}: Awareness, Carrying, 
Detection, Disarm Traps, Dual Wield, Escape, Hiding, Hunting, Jumping, 
Spelunking, Stealth, Thievery, Short Blades, Daggers, Thrown Daggers.

\item {\bf Warriors} include soldiers, barbarians, nomads, and sell-swords.
Some learn their skills by military training, while others learn simply 
through survival.  Coming from many walks of life, they all have skill in 
close combat.  \textbf{Initial Skills}: Blind Fighting, Boating, Bowyer,
Carrying, Combat, Detection, Fletchery, Hunting, Intimidation, Swimming,
Axes, Bludgeons, Daggers, Rods and Staves, Spears, Unarmed, Whips, Bows,
Rocks.

\item {\bf Witchlings} practice a primordial, chaotic magic.  From a young 
age, they find themselves able to channel the latent energies of the 
world.  Misunderstood and often marginalized, Witchlings often live apart 
from society, separated by the blessing and curse of their abilities.
\textbf{Initial Skills}: Awareness, Brewing, Cantrips, Detection, Foraging,
Forest Lore, Herbalism, Literacy, Lore, Magic, Marsh Lore, Scribing,
Swimming, Whips, Primordial Magic.

\item {\bf Wizards} are students of the arcane.  Often apprenticing at a 
young age, they spend years reading ancient tomes, learning spellcraft, 
brewing potions, and imbuing wands and staves with magical powers.  They 
are often seen in the company of some sort of familiar.
\textbf{Initial Skills}: Awareness, Brewing, Cantrips, Detection,
Herbalism, Literacy, Lore, Magic, Papercraft, Scribing, Wandcraft, Weaving, 
Daggers, Arcane Magic.
\end{itemize}

\subsection{Deities}

There are nine deities within the world of {\it Shadow of the Wyrm}, 
collectively referred to simply as, ``the Nine''.  Though the motives and 
actions of the divine can never be neatly categorized by mortal men, it is 
believed that three are largely good, three remain neutral, and three are 
deeply evil.

\subsubsection{The Good}

\begin{itemize}
\item Empress of the Heavens, {\bf Celeste} created the universe, crafting 
the galaxies and stars.  She watches the world from high above, where 
centuries pass like seconds.  Her domains are magic and creation.  
Throughout the existence of the universe, she has waged countless battles 
against the horrors of Sceadugenga.  Celeste grants her worshippers +2 
Intelligence.

\item Appearing to his followers as a great knight armoured in shining
plate and wielding a flaming blade, {\bf Aurelion} governs strength, 
chivalry, and honourable combat.  He is husband of The Lady, and is in 
constant struggle with the forces of black Urgoth.  Followers of Aurelion 
gain +2 Strength.

\item Wife of Aurelion, {\bf The Lady} is seen in the form of an 
impossibly beautiful and radiant woman.  She loves life, love, light, and 
music.  The Lady is the kindest and gentlest of the pantheon, and bestows 
+2 Charisma to her worshippers.
\end{itemize}

\subsubsection{The Neutral}

\begin{itemize}
\item After Celeste created the heavens, {\bf Vedere} created the world 
itself.  It was by his will that the mountains were raised.  The forests 
and grasslands are his, as are the lakes and seas, the deserts and 
marshes.  Itinerant wanderers and those who live off the land are often 
worshippers of Vedere, who protects his followers by granting +1 Health.

\item Once Vedere created the world, he made {\bf Voros} its protector.  
Voros lives deep within the molten core, a monstrous red wyrm, breathing 
gases and lava from his massive jaws.  Voros is often worshipped by the 
dwarves, and others who live deep within the earth.  Those who revere him 
receive +2 Strength.

\item Appearing as a cloaked figure accompanied by a crow, 
{\bf The Trickster} wanders the world of men, bending probability wherever 
he goes.  He acts as an agent of neutrality.  By his deeds, The Trickster 
balances the gains of order and chaos.  His blessing grants +2 Agility.
\end{itemize}

\subsubsection{The Evil}

\begin{itemize}
\item A withered, twisted figure, {\bf Shiver} takes the form of a bent and
haggard crone.  She comes in winter, on cold winds: her presence casts a 
pall over the landscape, a deathly chill that cannot be lifted.  To her 
followers, she grants +2 Willpower. 

\item The Black Ogre, {\bf Urgoth}, feeds off anger, rage, and hate.  He 
seeks the destruction of all things good and holy, and leads his hordes of 
chaos in an ongoing struggle against Aurelion's forces.  Those loyal to 
Urgoth receive +2 Strength.

\item A teeming black horror as old as time itself, {\bf Sceadugenga} lurks
in the blackest corners of the universe, held back by the power of the rest
of the pantheon.  He seeks nothing less than the destruction of all 
creation, culminating with the deaths of all the other gods.  His name is 
considered ill even to speak.  His few followers practice in utmost 
secrecy, and can be identified by a bleeding black mark on their 
foreheads.  Sceadugenga's minions receive +2 Intelligence.

\end{itemize}

It is said that, upon hearing the prayers of the most pious and devout,
that the worshipped deity may decide to crown that mortal as a holy
champion, providing fortification against damage and bestowing a gift
of great power.

\subsection{Alignment}

The three alignments are Good, Neutral, and Evil.  These represent sets
of values that the ancient philosophers formalized in an attempt to
categorize all creation.

\begin{itemize}
\item {\bf Good} and the concept of ``goodness'' encompass those who 
treat others well; who have a respect for life and dignity, and those
around them; and who will attempt to help others, even at a cost to
themselves.

\item {\bf Evil} creatures range from the narcissistic and self-centered 
to the diabolical.  Those who are evil lack respect for others, putting 
their own interests at the fore, and have few compunctions about harming 
others to get their own way; indeed, for some, maiming, torturing, and 
killing are the whole of their interests.

\item {\bf Neutral}, or unaligned, refers to those creatures and people
who stand outside the neat dichotomy of good and evil.  Some are mindless
and unthinking: slimes, low animals, and so on, who are incapable of making
moral decisions and live in a state of nature.  Others are intelligent
and rational creatures who make a conscientous decision to live apart from
good and evil, making moral decisions only as they must.  The Neutral
alignment describes all of these.

\end{itemize}

A character's initial alignment is determined by the deity selected at
character creation.

\subsection{Statistics}

A character's statistics influence many things: its ability to land or
dodge blows, its hardiness, its ability to deal damage or learn spells,
and many other things.  Statistics are displayed at the bottom of the
screen

\subsubsection{Primary Statistics}

There are seven primary statistics: Strength, Dexterity, Agility, Health,
Intelligence, Willpower, and Charisma.  These are all displayed on the
status lines at the bottom of the screen while you are playing the game.  
These statistics affect different calculations and outcomes within the 
game, and are described below.

\begin{itemize}
\item {\bf Strength} (Str) determines how strong a character is.  It 
affects weapon damage, as well as the ability to hit with large weapons 
such as great swords and huge hammers.

\item {\bf Dexterity} (Dex) represents a character's physical control, and 
is used to determine the ability to hit with most melee and ranged weapons.

\item {\bf Agility} (Agi) is a character's nimbleness, and allows it to 
more easily dodge attacks and other dangers.

\item {\bf Health} (Hea) is a measure of a character's toughness or 
hardiness.  It affects how much damage the character can take before dying,
as well as helping to resist certain unwanted statuses.

\item {\bf Intelligence} (Int) affects the ability of a character to 
successfully learn spells, and the number of spells that can be cast 
before exhaustion.

\item {\bf Willpower} (Will) also affects the ability of a character to 
learn certain spells.  It also has an impact on the number of spells that 
the character can cast before exhaustion, but to a lesser degree than
Intelligence.

\item {\bf Charisma} (Cha) determines how easily the character sways or
influences others, and also impacts whether monsters are intimidated by 
you.

The primary statistics can increase through the actions taken by the
player.  For instance, carrying heavy loads or attacking successfully with
a heavy weapon can help to increase Strength.  There is also a trainer
to be found who can help increase these statistics --- for a fee, of
course.

\end{itemize}
\subsubsection{Secondary Statistics}

Secondary statistics are much more focused than primary statistics, and
tend to have much more specific uses.  These statistics can be acquired
from armour and spells, or can be based on race and class.  These
statistics are also displayed on the status lines at the bottom of the
screen.

\begin{itemize}
\item {\bf Evade} (Ev) is used to determine whether a creature successfully
avoids an incoming attack.

\item {\bf Soak} (Sk) reduces the amount of damage on a successful attack,
removing one point of damage per point of Soak.

\item {\bf Speed} (Sp) determines how quickly a character can attack -- the
lower the Speed score, the sooner the character can act again.

\item {\bf Hit Points} (HP) represents a character's capacity for taking 
damage.  When a character's hit points are reduced to 0 or lower, the 
character is dead.

\item {\bf Arcana Points} (AP) are a character's ability to cast spells.  
Each spell has a particular arcana point cost associated with it, so the 
higher this score, the more spells the creature can cast before exhaustion.
\end{itemize}

\subsection{Skills}

\subsubsection{General}
\begin{itemize}
\item {\bf Archery}: Skill and damage with ranged weapons.
\item {\bf Awareness}: React quickly to incoming projectiles and magic.
\item {\bf Bargaining}: Buy for less, sell for more.
\item {\bf Beastmastery}: Tame wild creatures.
\item {\bf Blindfighting}: Fight and dodge effectively without sight. 
\item {\bf Boating}$\dagger$: Skill navigating on the water.
\item {\bf Bowyer}: Craft bows and crossbows.
\item {\bf Brewing}$\dagger$: Brew magic potions and moonshine. 
\item {\bf Carrying}: Carry more and heavier items.
\item {\bf Combat}: Skill and damage with melee weapons, and in techniques such as counter-striking.
\item {\bf Crafting}: Skill at generally creating things: magical items, skins, and many other things.
\item {\bf Desert Lore}$\dagger$: Knowledge of deserts and dunes. 
\item {\bf Detection}: Sense the unseen, such as nearby traps or creatures, both actively and passively.
\item {\bf Disarm Traps}: Disassemble dangerous traps. 
\item {\bf Dual Wield}: Fight effectively with two weapons. 
\item {\bf Dungeoneering}: Familiarity with dungeons and loot allows more items to be found, and fewer to break.
\item {\bf Escape}: Flee effectively and squeeze through tight spaces.
\item {\bf Fishing}: Catch fish.
\item {\bf Fletchery}: Create ammunition for ranged weapons.
\item {\bf Foraging}: Find more food in the wilderness.
\item {\bf Forest Lore}$\dagger$: Knowledge of the woods and forests. 
\item {\bf Herbalism}: Forage for useful herbs. 
\item {\bf Hiding}: Remain unseen from hostile foes.
\item {\bf Hunting}: Effectively hunt animals large and small. 
\item {\bf Intimidation}: Make your foes tremble and quiver. 
\item {\bf Jeweler}: Create and improve rings and amulets. 
\item {\bf Jumping}$\dagger$: Leap over obstacles and foes.
\item {\bf Leadership}: Command allies effectively and gain experience from their success. 
\item {\bf Literacy}: The ability to read scrolls and books.
\item {\bf Lore}: Knowledge of an item's blessed/uncursed/cursed status on pickup.
\item {\bf Magic}: Learn spells more effectively. 
\item {\bf Marsh Lore}$\dagger$: Knowledge about swamps and fens. 
\item {\bf Medicine}: Naturally heal wounds faster, and avoid poison, stoning, and other negative effects.
\item {\bf Mountain Lore}$\dagger$: Knowledge of mountain features. 
\item {\bf Mountaineering}: Skill at climbing tall peaks. 
\item {\bf Music}: Sway with songs and instruments. 
\item {\bf Night Sight}: Ability to see in low-light conditions. 
\item {\bf Oceanography}$\dagger$: Knowledge of seas and oceans. 
\item {\bf Papercraft}: Creation of paper from natural materials. 
\item {\bf Religion}: Knowledge of the Nine and their ways. 
\item {\bf Scribing}: Creation of scrolls and books. 
\item {\bf Skinning}: Make useful skins from dead creatures. 
\item {\bf Smithing}: Create and improve melee weapons.
\item {\bf Spelunking}$\dagger$: Move through tight spaces with ease.
\item {\bf Stealth}: Move without being heard, and attack with surprise.
\item {\bf Swimming}: Swim through water without drowning.
\item {\bf Tanning}: Turn skins into useful armour.
\item {\bf Thievery}: Redistribute wealth towards yourself.
\item {\bf Wandcraft}: The ability to create wands and make the most of them.
\item {\bf Weaving}: Create cloaks and clothing.
\end{itemize}

$\dagger$ -- Not implemented yet.

\subsubsection{Melee Weapons}

All melee weapons can be categorized as one of the following weapon skills: {\bf Axes}, {\bf Short Blades}, {\bf Long Blades}, {\bf Bludgeons}, {\bf Daggers}, {\bf Rods and Staves}, {\bf Spears}, {\bf Unarmed}, {\bf Whips}, {\bf Exotic}.

\subsubsection{Ranged Weapons}

All ranged weapons can be categorized as one of the following ranged weapon skills: {\bf Thrown Axes}, {\bf Thrown Blades}, {\bf Thrown Bludgeons}, {\bf Bows}, {\bf Crossbows}, {\bf Thrown Daggers}, {\bf Rocks}, {\bf Slings}, {\bf Thrown Spears}, {\bf Exotic Ranged}.

\subsubsection{Magic}

There are five spheres of magic, which encompass all the magics of the world.  These are:

\begin{itemize}
\item {\bf Arcane}: Ancient powers harnessed by long-past civilizations,
written down in an ancient Runic language.  Arcane magic is broad, having
elemental, restorative, protective, and many other types of spells.  The
language used to record Arcane magic is incredibly old and complicated.  It
requires heavy study itself, and thus the secrets of the Arcane are known
only to a small few.
\item {\bf Divine}: The secrets of the Nine, written down by sages and
scholars in ancient, hide-bound tomes.  Divine magic is largely
protective and restorative, cloaking the caster in the light and darkness
of the Nine.  Though it is recorded in the Runic script, the words
themselves are those of the titans and archangels, gifted in dreams only
to the most devout and holy.
\item {\bf Mystic}: Mystic magic allows the seer to glimpse the future in
tiny glances, seeing things just before they happen: terrible afflictions,
petrification, blindness, dumbness.  Mystic magic, like Arcane, is written
in the ancient Runic script, though its secrets are difficult to understand
except for the rare folk with the gift for far sight.
\item {\bf Primordial}: Primordial magic, often referred to simply as
``shadow magic'', harnesses the raw chaotic powers of the universe.  It is 
not written down; rather, it is wielded instinctively.  Primordial magic is
deeply offensive, allowing its user to unleash inky black flames and roiling
chaos.
\item {\bf Cantrips}: Minor magical spells.  Cantrips encompass a broad
spectrum of spells, from the attacking to the healing.  They are relatively
easy to learn, not being written in Old Runic, but they are also less
effective than their counterparts in the other spheres of magic, and more
expensive to cast.

\end{itemize}

Primordial magic is deeply innate.  Those that learn its workings do so
automatically as they gain in power.  The remaining magics are
traditionally learned through the careful study of ancient spellbooks.

When casting magic, you select a spell from a screen that summarizes
the spells your character knows.  This screen contains:

\begin{itemize}
\item The name and sphere of the spell
\item The number of castings available before the spell is forgotten
\item Range, shape, and radius details
\item The cost in Arcana Points
\item The spell's effectiveness rating
\end{itemize}

A sample screen might look like:

\begin{verbatim}
[a] Cross of Flame (Arcane) [#:20, R:3Cr, AP:10, +0]
[b] Dragon Breath (Arcane) [#:10, R:3C, AP:7, +0]
[c] Lightning Bolt (Arcane) [#:15, R:4RB, AP:6, +0]
[d] Wreath of Fire (Arcane) [#:22, R:2Ba, AP:5, +0]
[e] Absolution (Divine) [#:40, R:6B2, AP:50, +0]
[f] Blink (Cantrips) [#:30, R:0TS, AP:3, +0]
[g] Spirit Bolt (Cantrips) [#:44, R:4RB, AP:30, +2]
\end{verbatim}

The range, shape, and radius is encoded in a value like ``R:3Cr''.  This
means that the spell has a range of 3 and is cross-shaped.  A value of
``R:6B2'' means that the spell has range 6, beam-shaped, with radius 2
(indicating a wider beam).  The following table summarizes the spell
shapes.

\begin{table}[h]
\begin{tabular}{lll}
{\bf Abbreviation} & {\bf Shape} & {\bf Description} \\
B & Beam & Single-directional, non-reflective beam \\
Ba & Ball & Outward-spreading ball \\
C & Cone & Cone \\
Cr & Cross & Beams in the cardinal directions \\
RB & Reflective Beam & Single-directional, reflective beam \\
St & Storm & Random tiles around the caster \\
TS & Target Self & Targets the caster \\
\end{tabular}
\end{table}

Radius only affects beam-shaped spells.  A radius of 1 is a typical
beam spell that travels along in a line.  Each increment to the radius
adds a tile above and below that, so that radius 2 creates a beam along
3 lines, radius 3 a beam along 5 lines, and so on.

\subsection{Resistances}

In {\it Shadow of the Wyrm}, there are many different kinds of dangers.  
Bandits carry cheap knives and swords.  Ogres and trolls wield massive
wooden clubs.  Dragons breath infernal flames, and wizards cast ancient 
spells.  These are all represented by different damage types, and every 
creature is affected differently, based on race, class, current equipment, 
and the currently in-force spells.

All creatures in the game, not just the player's character, have a set of 
resistances.  Some creatures may be almost invincible when attacked with
certain damage types, and may be deeply vulnerable to others.

The different damage types follow, with an example of how they are used
in-game.

\begin{itemize}
\item {\bf Slash}: swords, axes, etc.
\item {\bf Pierce}: daggers, spears, etc.
\item {\bf Pound}: clubs, maces, etc.
\item {\bf Heat}: fire-based spells, dragon breath, fiery weapons, etc.
\item {\bf Cold}: ice-based spells, certain undead/astral creatures, etc.
\item {\bf Acid}: certain jellies and slimes, insects, demons, etc.
\item {\bf Poison}: certain snakes and reptiles, sea creatures, demons, etc.
\item {\bf Holy}: divine magics, holy creatures, etc.
\item {\bf Shadow}: primordial magics, certain undead/astral creatures, etc.
\item {\bf Arcane}: arcane magics, certain demons, etc.
\item {\bf Lightning}: lightning-based spells, certain holy creatures, etc.
\end{itemize}

The list above is intentionally incomplete -- there are many different
creatures scattered throughout the world, and a well-prepared character 
should have preparations in place for many different possibilities, not 
just the most common.

\subsection{Status Ailments}

Each of the above damage types can also have a lingering effect on the
targeted creature.  Status ailments affect the creature in a variety of
way, and may range from an inconvenience, to downright dangerous or
deadly.  After each successful hit, a status ailment may be applied.  The
chance for this is modified by the targeted creature's resistance to the
damage type, as well as potentially other modifiers.

\begin{itemize}
\item {\bf Blinded (Fire)}: cannot see any surroundings.  Severe combat 
penalties.
\item {\bf Bloodied (Slash)}: difficult to focus accurately on combat.  
\item {\bf Disfigured (Acid)}: randomly reduces a number of statistics.
\item {\bf Exposed (Pierce)}: knocked off-balanced, evasion becomes 
difficult.  
\item {\bf Mute (Holy)}: cannot speak, and therefore chat, cast spells, 
etc.
\item {\bf Paralysis (Lightning)}: cannot move or act.
\item {\bf Poison (Poison)}: damage over time until the poison is cured.
\item {\bf Slow (Cold)}: moves and acts at a much slower rate.
\item {\bf Spellbound (Arcane)}: can act, but cannot move.
\item {\bf Stone (Shadow)}: eventually, become petrified into a statue 
unless the corruption is cured.
\item {\bf Stunned (Pound)}: while staggering, moving or attacking in the 
desired direction is unlikely.
\end{itemize}

